% Facultad de Ingeniería, Universidad de Buenos Aires
% 75.10 Técnicas de diseño

\documentclass[a4paper,10pt, notitlepage]{article}
\usepackage[utf8]{inputenc} 
\usepackage[spanish]{babel} 
\usepackage{listings}
\usepackage{color} 

\usepackage{graphicx}
\usepackage{verbatim}

\pagestyle{plain} %o pagestyle{plain} , valor defecto   {headings}
% define the title
\title{\textbf{75.10 Técnicas de Diseño, TP Grupal}}
% \author{Rodriguez Genaro, Leandro \and Padron: 92098 \\ 
\author{Awad, Lucas \and Choque, Javier \and López, Federico \and Rodriguez, Leandro}
\date{}


\begin{document}
% generates the title
\maketitle

\thispagestyle{empty}

\newpage

\section{Analisis Tp}

\subsection{Pros:}
\begin{itemize}
\item Interfaz de usuario muy amigable. 
\item El modelo estaba clausurado ante cambios (no asi la vista) por lo que era relativamente sencillo aplicar el cambio pedido 
sobre el modelo, asi como tambien agregar nuevos tipos de ofertas.
\end{itemize}

\subsection{Contras:}
\begin{itemize}
\item No pasaba todos los test unitarios, ademas de que tenía pocos.
\item Los productos se creaban hardcodeados en una clase.
\item La vista no estaba bien separada del modelo. Prueba de esto, por ejemplo, es que los test, imprimen por consola la compra 
como cuando se corre el programa con la interfaz gráfica.
\item Modelaron los descuentos como productos, en los que el nombre describe el tipo de descuento que es.
\item Codigo poco legible. En algunas clases se utilizaban variables nombradas con una sola letra, lo cual no nos daba idea de que 
hacía, por lo que teníamos que leer el codigo para intentar entender para que servía. Tambien se definían métodos adentro de otros 
métodos, haciendo incómoda la lectura del código.
\item Malas prácticas de código. Mas allá de lo antes mencionado, un ejemplo podría ser el siguiente: Un metodo declarado en una 
clase, retornaba como valor una clase hijo. Claro está que este método devolvía null en esta clase, y en la clase hijo tenía 
definido otro comportamiento.
\item Dependencia de clases externas. Por ejemplo, para manejo de fechas de las ofertas, usaron directo la clase Calendar de java, 
en lugar de usar una clase propia que encapsule el acceso a la clase Calendar.
\end{itemize}

\end{document}
